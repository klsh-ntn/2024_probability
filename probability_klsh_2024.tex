% arara: xelatex: {shell: yes}
% %arara: biber
% %arara: xelatex: {shell: yes}
% %arara: xelatex: {shell: yes}

\documentclass[12pt]{article}
\usepackage{libertine}

\usepackage{hyperref} % гиперссылки

\usepackage{tikz} % картинки в tikz
\usetikzlibrary{arrows.meta} % tikz-прибамбас для рисовки стрелочек подлиннее

\usepackage{microtype} % свешивание пунктуации

\usepackage{array} % для столбцов фиксированной ширины

\usepackage{indentfirst} % отступ в первом параграфе

\usepackage{sectsty} % для центрирования названий частей
\allsectionsfont{\centering}

\usepackage{amsmath} % куча стандартных математических плюшек
\usepackage{amssymb} % символы
\usepackage{amsthm} % теоремки

\usepackage{comment} % добавление длинных комментариев

\usepackage[top=1cm, left=1.2cm, right=1.2cm, bottom=2cm]{geometry} % размер текста на странице

\usepackage{lastpage} % чтобы узнать номер последней страницы

\usepackage{enumitem} % дополнительные плюшки для списков
%  например \begin{enumerate}[resume] позволяет продолжить нумерацию в новом списке

\usepackage{caption} % что-то делает с подписями рисунков :)

\usepackage{qcircuit} % для рисовки квантовых диаграмм
\usepackage{physics} % бракеты

\usepackage{answers} % разделение условий и ответов в упражнениях


\usepackage{fancyhdr} % весёлые колонтитулы
\pagestyle{fancy}
\lhead{Теория невероятностей}
\chead{}
\rhead{КЛШ-2024}
\lfoot{}
\cfoot{}
\rfoot{\thepage/\pageref{LastPage}}
\renewcommand{\headrulewidth}{0.4pt}
\renewcommand{\footrulewidth}{0.4pt}



\usepackage{todonotes} % для вставки в документ заметок о том, что осталось сделать
% \todo{Здесь надо коэффициенты исправить}
% \missingfigure{Здесь будет Последний день Помпеи}
% \listoftodos — печатает все поставленные \todo'шки



\usepackage{booktabs} % красивые таблицы
% заповеди из докупентации:
% 1. Не используйте вертикальные линни
% 2. Не используйте двойные линии
% 3. Единицы измерения - в шапку таблицы
% 4. Не сокращайте .1 вместо 0.1
% 5. Повторяющееся значение повторяйте, а не говорите "то же"



\usepackage{fontspec} % что-то про шрифты?
\usepackage{polyglossia} % русификация xelatex

\setmainlanguage{russian}
\setotherlanguages{english}

% download "Linux Libertine" fonts:
% http://www.linuxlibertine.org/index.php?id=91&L=1
% \setmainfont{Linux Libertine O} % or Helvetica, Arial, Cambria
% why do we need \newfontfamily:
% http://tex.stackexchange.com/questions/91507/
% \newfontfamily{\cyrillicfonttt}{Linux Libertine O}

\AddEnumerateCounter{\asbuk}{\russian@alph}{щ} % для списков с русскими буквами
\setlist[enumerate, 2]{label=\asbuk*),ref=\asbuk*}

%% эконометрические сокращения
\DeclareMathOperator{\Cov}{Cov}
\DeclareMathOperator{\Arg}{Arg}
\DeclareMathOperator{\Corr}{Corr}
\DeclareMathOperator{\Var}{Var}
\DeclareMathOperator{\E}{\mathbb{E}}
\newcommand \hVar{\widehat{\Var}}
\newcommand \hCorr{\widehat{\Corr}}
\newcommand \hCov{\widehat{\Cov}}
\newcommand{\cN}{\mathcal{N}}

\newcommand{\addtag}[1]{}

\let\P\relax
\DeclareMathOperator{\P}{\mathbb{P}}

\usepackage{multicol}

\usepackage[bibencoding = auto,
backend = biber,
sorting = none,
style=alphabetic]{biblatex}

\addbibresource{forecast_everything.bib}



% делаем короче интервал в списках
\setlength{\itemsep}{0cm}
\setlength{\parskip}{0cm}
\setlength{\parsep}{0cm}




\Newassociation{sol}{solution}{solution_file}
% sol --- имя окружения внутри задач
% solution --- имя окружения внутри solution_file
% solution_file --- имя файла в который будет идти запись решений
% можно изменить далее по ходу
\Opensolutionfile{solution_file}[all_solutions]
% в квадратных скобках фактическое имя файла

% магия для автоматических гиперссылок задача-решение
\newlist{myenum}{enumerate}{3}
% \newcounter{problem}[chapter] % нумерация задач внутри глав
\newcounter{problem}[section]

\newenvironment{problem}%
{%
\refstepcounter{problem}%
%  hyperlink to solution
     \hypertarget{problem:{\thesection.\theproblem}}{} % нумерация внутри глав
     % \hypertarget{problem:{\theproblem}}{}
     \Writetofile{solution_file}{\protect\hypertarget{soln:\thesection.\theproblem}{}}
     %\Writetofile{solution_file}{\protect\hypertarget{soln:\theproblem}{}}
     \begin{myenum}[label=\bfseries\protect\hyperlink{soln:\thesection.\theproblem}{\thesection.\theproblem},ref=\thesection.\theproblem]
     % \begin{myenum}[label=\bfseries\protect\hyperlink{soln:\theproblem}{\theproblem},ref=\theproblem]
     \item%
    }%
    {%
    \end{myenum}}
% для гиперссылок обратно надо переопределять окружение
% это происходит непосредственно перед подключением файла с решениями



\theoremstyle{definition}
\newtheorem{definition}{Определение}



\begin{document}

\tableofcontents{}

\section*{Анонс}
...

\newpage
\setcounter{section}{0}


\section{Деревья}

% Подборка задач для распечатки, умещается на один лист, разумно распечатать компактно, 
% упаковав две А5 страницы на один лист А4 и потом разрезать.

\newpage % для выделения отдельной страницы

\newcommand{\dayone}{
\begin{enumerate}
  \item  Неправильную монетку с вероятностью «орла» равной $0.8$ подбрасывают до выпадения последовательности $HT$, но не более четырёх раз.
  \begin{enumerate}
    \item Какова вероятность того, что последовательность $HT$ выпадет?
    \item Какова вероятность того, что хотя бы раз выпадет последовательность $HHH$?
    \item Какова вероятность того, что эксперимент закончится ровно за три броска?
    \item Сколько в среднем бросков получается в этом эксперименте?
  \end{enumerate}

  \item  Неправильную монетку с вероятностью «орла» равной $0.8$ подбрасывают до первого «орла».
  \begin{enumerate}
    \item Какова вероятность чётного числа бросков? 
    \item Чему равно среднее количество подбрасываний?
    \item Чему равно среднее количество орлов?
    \item Чему равно среднее количество решек?
  \end{enumerate}

  \item Подбрасываем правильную монетку бесконечное количество раз. 
  \begin{enumerate}
    \item Сколько в среднем ждать до появления последовательности HTT? 
    \item Сколько в среднем ждать до появления последовательности THT? 
    \item Какова вероятность того, что последовательность HTT будет выкинута раньше THT?
    \item Какова вероятность того, что последовательность HTT будет выкинута раньше HTH?
    \item Сколько в среднем ждать до появления HTT или THT?
  \end{enumerate}
    
\item Илья Муромец стоит на развилке у камня и раздумывает о своём будущем. 
От камня начинаются три дороги. 
Каждая из дорог оканчивается камнем.
И от каждого камня начинаются ещё три дороги. 
И каждые те три дороги оканчиваются камнем\ldots{ } 
И так далее до бесконечности. 
Каждую дорогу независимо от других с вероятностью одна треть охраняет трёхголовый Змей Горыныч. 

\begin{enumerate}
  \item Какова вероятность того, что ИМ рано или поздно встретит ЗГ, если выбирает дороги равновероятно на своём бесконечном жизненном пути?
  \item У ИМ есть волшебная карта, на которой показаны все дороги и все ЗГ. 
  Какова вероятность того, что ИМ \textit{сможет найти} путь, избегающий встречи с ЗГ?
\end{enumerate}

\item В каждой вершине треугольника по ёжику. Каждую минуту с вероятностью $0.5$ каждый ежик
независимо от других двигается по часовой стрелке, с вероятностью
$0.5$ — против часовой стрелки.
Обозначим $T$ — время до встречи всех ежей в одной вершине.

Найдите $\P(T=1)$, $\P(T=2)$, $\P(T=3)$, $\E(T)$.

\item Саша и Маша поженились и решили, что будут заводить новых детей до тех пор,
пока в их семье не будут дети обоих полов. Обозначим $X$ — количество детей в их семье.
Найдите $\P(X=4)$, $\E(X)$.

\item Саша и Маша по очереди подбрасывают кубик\addtag{кубик} до первой шестёрки.
Посуду будет мыть тот, кто первым выбросит шестерку.
Маша бросает кубик первой.

Какова вероятность того, что посуду будет мыть Маша?
Сколько в среднем раз они будут бросать кубик\addtag{кубик}?
\end{enumerate}
}

Занятие 1. 

Пришло 20 человек. 
По ходу решения задачи ввели обозначения $\P(A)$ для события,
$\E(L)$ для случайной величины и сокращение $H^3(HT)^2$ для $HHHHTHT$.
Мотивировали умножение вероятностей на дереве и формулу для ожидания через большое количество экспериментов. 
Одна из школьниц подошла и сказала: 
«Я всегда думала, что формула для математического ожидания — это просто люди так договорились.
А оказывается, она логично определяется».
Решили первую задачу. Во второй задаче решили первый пункт двумя способами. 
Через сворачивание суммы бесконечно убывающей геометрической прогрессии и методом первого шага. 
Обратили внимание, что поиск самоподобия в сумме важнее, чем готовая формула. 

Занятие 2. 

Пришло 20 человек. 
Решили вторую задачу пункт (б) через вычисление суммы и методом первого шага.
Дорешали вторую задачу через балансовое уравнение $N_H + N_T = L$ и методом первого шага. 
В третьей задаче решили первые два пункта методом первого шага. 
Один семиклассник впервые решал системы из трех уравнений, 
но вполне бодро справился сам после подсказки, 
что сначала каждое уравнение разумно упростить, а затем стоит избавиться от одной из неизвестных. 

Занятие 3. 

Пришло 20 человек. 
Решили задачу про вероятность HTT vs THT. 
Двумя способами: через рекуррентное отношение на вероятности и через производящии функции многих переменных. 
Это просто огонь! Девятиклассники поняли. Даже указали, как появляется автоматически ожидание из множества неоконченных позиций. 

Задание 4. 

Пришло 20 человек. 
Решили задачу про ЗГ и даже немного обсудили отбор корней. 
Я рассказал идею введения параметра и графика зависимости корней от параметра. 
В данной задаче следует отбирать траекторию. 
Но сами графики траекторий мы не рисовали, просто нашли несколько точек на графике. 
Я упростил описание ситуации, сказав, что надо отобрать траекторию, 
а в задаче в реальности происходит перескок с одной траектории на другую. 
Далее дошли до составления уравнения на задачу про игру с деньгами на кону. 
Это могло бы быть хорошее компьютерное упражнение!
Можно бы даже ещё задачку со стратегией с компьютерным подбором корней. 


\newcommand{\partB}{
\begin{enumerate}
  \item Изначально на кону лежит ноль рублей. Кубик подкидывают неограниченное число раз.
  Если на кубике выпадает 1, то ты получаешь 1 рубль на руки и игра продолжается. 
  Если на кубике\addtag{кубик} выпадает 2 или 3, то соответствующее количество рублей добавляют на кон и игра продолжается.
  Если выпадает 4 или 5, то игра оканчивается и ты получаешь сумму, лежащую на кону.
  Если выпадает 6, то игра оканчивается, а ты не получаешь ничего. 
  \begin{enumerate}
  \item Какова вероятность того, что игра рано или поздно закончится выпадением 6-ки?
  \item Какова ожидаемая продолжительность игры?
  \item Чему равен ожидаемый выигрыш?
  \item Чему равен ожидаемый выигрыш, если изначально на кону лежит 100 рублей?
  \item Изменим изначальное условие: если выпадает 5, то сумма на кону сгорает, а игра продолжается.
  Чему будет равен средний выигрыш в новую игру?
  \end{enumerate}
  \item У тети Маши — двое детей, один старше другого. Предположим, что вероятности рождения мальчика и девочки равны и не зависят от дня недели, а пол первого и второго ребенка независимы. Для каждой из четырех ситуаций найдите условную вероятность того, что у тёти Маши есть дети обоих полов.
\begin{enumerate}
\item Известно, что хотя бы один ребенок — мальчик.
\item Тетя Маша наугад выбирает одного своего
ребенка и посылает к тете Оле, вернуть учебник по теории
вероятностей. Это оказывается мальчик.
\item Известно, что старший ребенок — мальчик.
\item На вопрос: «А правда ли тетя Маша, что у вас есть сын, родившийся в пятницу?» тётя Маша ответила: «Да».
\end{enumerate}
\item У Ивана Грозного $10$ бояр. 
Каждый боярин берёт мзду независимо от других с вероятностью $1/2$.

\begin{enumerate}
  \item Какова вероятность того, что все бояре берут мзду, если случайно выбранный боярин берёт мзду?
  \item Какова вероятность того, что все бояре берут мзду, если хотя бы один из бояр берёт мзду?
\end{enumerate}
\item Есть три одинаковых с виду закрытых дверей. За одной из них — автомобиль, за остальными — по козе.
Маша выбирает одну из дверей.
Ведущий шоу, чтобы поддержать интригу, не открывает сразу ни выбранную Машей дверь,
ни дверь с автомобилем. 
Из возможных для поддержания интриги вариантов ведущий выбирает равновероятно.
Игра началась, Маша выбрала дверь номер 2, затем ведущий открыл дверь номер 3. 
Зрители в напряжении. 
И в этот момент ведущий предлагает Маше изменить выбор двери.

Имеет ли смысл Маше изменить свой выбор?


\item Аня играет в карты. Она получила 6 случайных карт из колоды в 52 карты.
\begin{enumerate}
 \item Какова вероятность того, что у Ани как минимум два туза?
 \item Какова вероятность того, что у Ани как минимум два туза, если у неё есть хотя бы один туз?
 \item Какова вероятность того, что у Ани как минимум два туза, если у неё есть туз пик?
 \item Какова вероятность того, что у Ани как минимум два туза, если у неё случайно выпала карта и это оказался туз?
 \item Какова вероятность того, что у Ани как минимум два туза, если у неё случайно выпала карта и это оказался туз пик?
\end{enumerate}
\end{enumerate}


}




\newpage
\thispagestyle{empty}
\dayone
\newpage
\thispagestyle{empty}



\dayone

\newpage
\thispagestyle{empty}
\partB
\newpage
\thispagestyle{empty}
\partB



% \newpage % для выделения отдельной страницы

Замечания:

Вводим обозначения $\Omega$, $A$, $\P(A)$, $\E(X)$.

Разница: событие и случайная величина.

При поиске вероятностей использовали степени букв! 
Это хорошая идея, чтобы потом писать производящие функции!

Например,
\[
\P(THHHHTT) = \P(TH^4T^2)  
\]





Роберт Адлер нажимает на кнопку «Вкл/Выкл» на пульте дистанционного
управления телевизором. Изначально телевизор включён. Батарейки
у пульта садятся, поэтому в первый раз кнопка срабатывает с вероятностью
1/2, а далее вероятность срабатывания кнопки падает, причем падает совершенное непредсказуемым образом. 

\begin{enumerate}
  \item Какова вероятность того, что после 2022 нажатий телевизор окажется включён?
  \item Кто такой Роберт Адлер?
\end{enumerate}



Какова вероятность того, что у здесь собравшихся есть хотя бы одно совпадение по дням рождения?
% На турнире команд ФМТ одновременно в основном составе участвует 88 школьников. 
% Какова вероятность того, что у них есть хотя бы одно совпадение по дням рождения?
А если бы нас собралось 50 человек?


\section{Деревья и уравнения на ожидания}

Упражнение. Неправильную монетку с вероятностью «орла» равной $0.7$ подбрасывают до первого «орла».
Чему равно среднее количество подбрасываний?  Орлов? Решек? Какова вероятность чётного числа бросков? 

Ищем математическое ожидание. 

Через составление рекуррентного уравнения
\[
a = 0.7 \cdot 1 + 0.3 (1 + a).
\]
Через мысленное повторение большого количества экспериментов и подсчета, сколько бросков придется на одного достигнутого орла. 

Через нахождение таблички распределения и суммирования. 

Записали случайные величины количества бросков $N$ и количества решек $R$ как функции. 
Например, $N(HHT) = 3$ или $R(HHHHT)= 4$.

Вероятность для чётного бросков нашли только через суммирование (можно было уравнением).

И без формального определения ввели производящую функцию. 

Множество (событие):
\[
A = \{HT, HHHT, HHHHHT, \ldots\}
\]
Производящая функция (интересующий нас объект записанный как функция)
\[
g(H, T) = H\cdot T + H\cdot H\cdot H\cdot T + H^5T + \ldots  
\]
Вероятность
\[
\P(A) = g(0.3, 0.7) = 0.3 \cdot 0.7 + 0.3^3 0.7 + 0.3^5 0.7 + \ldots  
\]


Упражнение. Подбрасываем монетку бесконечное количество раз. 

Какова вероятность того, что последовательность HTT будет выкинута раньше THT?

Какова вероятность того, что последовательность TTH будет выкинута раньше THT?

\[
A = \{HTT \text{ выпадет раньше } THT\}, \quad  B = \{TTH \text{ выпадет раньше } THT\}
\]

Нарисовали дерево с упрощениями. Срезали «уши» и назвали этот метод «методом Ван-Гога».
На упрощенном дереве видно, что ситуация симметричная, поэтому $\P(A) = 0.5$.

Школьники в большинстве сами построили дерево для вычисления $\P(B)$. Оно уже не симметричное.
По нему составляем вместе уравнение на $b=\P(B)$:
\[
b = 0.5 + 0.25b  
\]
И получаем $b=2/3$. Замечаем чудо! Число букв одинаковое и оказывается важен их порядок!


О школьниках: было 15 человек, трое не знали, что такое геометрическая прогрессия, 
поэтому просто выводили сумму с помощью домножения и вычитания. 
Искали слагаемые с парой в двух суммой и одно слагаемое «одинокое» без пары. 

\section{Задача о ежах}

В каждой вершине треугольника по ёжику. Каждую минуту с вероятностью $0.5$ каждый ёжик
независимо от других двигается по часовой стрелке, с вероятностью
$0.5$ — против часовой стрелки.
Обозначим $T$ — время до встречи всех ёжиков в одной вершине для чаепития.

\begin{enumerate}
  \item Постройте схему возможных взаимных позиций и найдите вероятности перехода между позициями. 
  \item Найдите $\P(T=1)$, $\P(T=2)$, $\P(T=3)$, $\E(T)$.
  \item В момент каждого посещения позиции ежи получают по 100 шишек каждый. 
 Обозначим количество шишек, собранных ежами к моменту чаепития, буквой $R$. Найдите $\E(R)$.
 \item Найдите вероятность $\P(T\text{  — чётное})$.
\end{enumerate}

Удобно для наглядности обозначить вероятности перехода буквами, $\alpha$, $\beta$, $\gamma$, \ldots.
И в буквах даже $\P(T=4)$ легко выписать. 

Далее используем неожиданный трюк. Найти $\E(T)$ сразу сложно. 
Однако, легко составить систему на $\E(T \mid \text{ старт в }A)$, $\E(T \mid \text{ старт в }B)$, $\E(T \mid \text{ старт в }С)$.

Аналогичная система составляется для $\E(R)$ и для $\P(T — \text{ чётное})$.

\section{Условная вероятность}

Решили задачу про тётю Машу и двух детей и про вероятность быть больным при условии, что человек по тесту болен.

Что-то я, вероятно, не докрутил, кажется школьники не оч впечатлились. 


\section{Условная подборка}
% \newpage
\begin{enumerate}
  \item Имеется три монетки. Две «правильных» и одна — с
  «орлами» по обеим сторонам. Петя выбирает одну монетку наугад и
  подкидывает её два раза. Оба раза выпадает «орел». Какова
  условная вероятность того, что монетка «неправильная»?
\item   Два охотника одновременно выстрелили в одну утку. Первый попадает с
вероятностью 0.4, второй — с вероятностью 0.7 независимо от первого.
\begin{enumerate}
\item Какова вероятность того, что в утку попала ровно одна пуля?
\item  Какова условная вероятность того, что утка была убита первым
охотником, если в утку попала ровно одна пуля?
\end{enumerate}
\item Игрок получает 13 карт из колоды в 52 карты.
Какова вероятность, что у него как минимум два туза, если
известно, что у него есть хотя бы один туз?
Какова вероятность того, что у него как минимум два туза, если
известно, что у него есть туз пик?
\item У тети Маши — двое детей, один старше другого. Предположим, что вероятности рождения мальчика и девочки равны и не зависят от дня недели, а пол первого и второго ребенка независимы. Для каждой из четырех ситуаций найдите условную вероятность того, что у тёти Маши есть дети обоих полов.
\begin{enumerate}
\item Известно, что хотя бы один ребенок — мальчик.
\item Тетя Маша наугад выбирает одного своего
ребенка и посылает к тете Оле, вернуть учебник по теории
вероятностей. Это оказывается мальчик.
\item Известно, что старший ребенок — мальчик.
\item На вопрос: «А правда ли тетя Маша, что у вас есть сын, родившийся в пятницу?» тётя Маша ответила: «Да».
\end{enumerate}
\item У Ивана Грозного $n$ бояр. 
Каждый боярин берёт мзду независимо от других с вероятностью $1/2$.

\begin{enumerate}
  \item Какова вероятность того, что все бояре берут мзду, если случайно выбранный боярин берёт мзду?
  \item Какова вероятность того, что все бояре берут мзду, если хотя бы один из бояр берёт мзду?
\end{enumerate}
\item Есть пять закрытых дверей. За одной из них — автомобиль, за остальными — по козе.
Маша выбирает одну из дверей.
Ведущий шоу, чтобы поддержать интригу, не открывает сразу выбранную Машей дверь.
Сначала он открывает одну из дверей не выбранных Машей,
причем ради интриги ведущий не открывает сразу и дверь с автомобилем. Из возможных вариантов он выбирает равновероятно.
Допустим, ведущий открыл дверь номер 3. 
И в этот момент он предлагает Маше изменить ваш выбор двери.

Имеет ли смысл Маше изменить свой выбор?
\item Аня хватается за верёвку в форме окружности в произвольной точке.
Боря берёт мачете и с завязанными глазами разрубает
верёвку в двух случайных независимых местах. Аня забирает себе тот кусок,
за который держится. Боря забирает оставшийся кусок. Вся верёвка имеет единичную длину.
\begin{enumerate}
\item Какова вероятность того, что у Ани верёвка длиннее?
\item Какова вероятность того, что Ане досталось больше четверти веревки, если ей досталось меньше, чем Боре?
\end{enumerate}


\end{enumerate}

\section{Решаем условные вероятности}

Решили про монетку, про утку и охотников, с тузами без $C_n^k$ оказалось сложновато, но почти до конца дошли. 

% \newpage

\section{Немного про факториалы и шляпы}

Дорешали задачу про Аню и Борю, делящих верёвку на плоскости. Обратили внимание на парадокс: 
возможные события могут иметь нулевую вероятность. Аргументировал аналогией с площадью. 
Если на плоскости ничего не нарисовано, то площадь нарисованного равна нулю. Если на плоскости 
нарисован отрезок, идеально математический, то площадь нарисованного тоже равна нулю, хотя что-то уже нарисовано.
При первом знакомстве с задачей про Аню и Борю школьники давали ответ $1/2$, поэтому после решения с площадью 
на плоскости поменяли местами ход Бори и Ани, чтобы прокачать интуицию.  

Вспомнили, что такое факториал, большинство школьников знали. 
Вспомнили или узнали, как посчитать число способов надеть 3 зелёных 2 красных и 4 синих шляпы на 10 человек. 
На этот вопрос знали ответ только двое школьников. Вернулись и дорешали задачу про туза пик. 
Согласились, что ответ громоздкий, но всё же уже доступный к дорешиванию на калькуляторе. 


\section{Оптимизация}

\newpage
\begin{enumerate}
  \item Маша и Саша играют в быстрые шахматы. У них одинаковый класс игры и
  оба предпочитают играть белыми потому, что выигрывают белыми с вероятностью $0.7$.
  В ничью они никогда не играют.

  Партии играются до 10 побед. Первую партию Маша играет белыми.
  
  Она считает, что в следующей партии белыми должен играть тот,
  кто выиграл предыдущую партию. Саша считает, что ходить белыми нужно по очереди.

  При каком варианте правил у Маши больше шансы выиграть?

  \item Злобный Дракон поймал принцесс Настю и Сашу и посадил в разные башни.
  Перед каждой из принцесс Злобный Дракон подбрасывает один раз правильную монетку.
  А дальше даёт каждой из них шанс угадать, как выпала монетка у её подруги.
  Если хотя бы одна из принцесс угадает, то Злобный Дракон отпустит принцесс на волю.
  Если обе принцессы ошибутся, то они навсегда останутся у него в заточении.
  
  Подобная практика у Злобного Дракона исследователями была отмечена уже давно,
  поэтому принцессы имели достаточно времени договориться на случай вероятного похищения.
  
  Как следует поступать принцессам при подобных похищениях?

  \item Вася подкидывает кубик\index{кубик} до тех пор, пока на кубике не выпадет единица, или пока он сам не скажет «Стоп». 
  Вася получает столько рублей, сколько выпало на кубике при последнем броске. 
  Вася хочет максимизировать свой ожидаемый выигрыш.
  \begin{enumerate}
  \item Как выглядит оптимальная стратегия? Чему равен ожидаемый выигрыш при использовании оптимальной стратегии?
  \item Какова средняя продолжительность игры при использовании оптимальной стратегии?
  \item Как выглядит оптимальная стратегия и чему равен ожидаемый выигрыш, если за каждое
  подбрасывание Вася платит 35 копеек?
  \end{enumerate}

  \item Андрей Абрикосов, Борис Бананов и Вова Виноградов играют одной командой в игру.
  В комнате три закрытых внешне неотличимых коробки: с абрикосами, бананами и виноградом.
  Общаться после начала игры они не могут, но могут заранее договориться о стратегии.
  Они заходят в комнату по очереди.
  Каждый из них может открыть две коробки по своему выбору.
  Перед следующим игроком коробки закрываются. Если Андрей откроет коробку абрикосами,
  Борис — с бананами, а Вова — с виноградом, то они выигрывают.
  Если хотя бы один из игроков не найдёт свой фрукт, то их команда проигрывает.
  
Какова оптимальная стратегия?
  \item Три игрока решили стреляться ради самой красивой девушки и организуют труэль
  (дуэль для трёх игроков).  Игроки стреляют по очереди, $A$-$B$-$C$-$A$-\ldots.
  Вероятности попадания в выбранную цель равны $p_a=0.6$, $p_b=0.5$ и $p_c=0.4$, соответственно.
  Игра продолжается до определения единственного победителя,
  он и получает девушку в жёны.
  
  Как выглядит оптимальная стратегия каждого игрока?
  \item Роковая дама играет в азартную игру.
  Перед дамой хорошо перетасованная колода в 52 карты.
  Дама открывает карты одну за одной. В любой момент дама может
  сделать пророчество «Следующая карта будет дамой».
  Если пророчество сбывается, то дама получает 100 рублей.
  
 Какова оптимальная стратегия дамы?
  
  

  % \item В ювелирном магазине широко открыта дверь, а на открытой витрине лежат 150 колец и 420 кулончиков. 
  % Пока продавщица болтает по телефону проказница-сорока залетает в магазин хватает по одному украшению наугад и относит к себе в гнездо
  % до полного опустошения магазина.

  % Какова вероятность того, что в какой-то момент времени в гнезде сороки колец будет столько же, сколько кулончиков?
  % \item Абу Али Хусейн ибн Абдуллах ибн аль-Хасан ибн Али ибн Сина выбирает равномерно и независимо 10 точек на отрезке $[0;30]$.

  % Какова вероятность того, что между любыми точками будет расстояние не меньше единицы?
\end{enumerate}
\newpage
\section{Just for fun}
\newpage

\begin{enumerate}
  \item В самолёте $100$ мест и все билеты проданы.
  Первой в очереди на посадку стоит Сумасшедшая Старушка,
  она очень переживает, что ей не хватит места.
  Сумасшедшая Старушка врывается в самолёт
  и несмотря на номер по билету садится на случайно выбираемое место.
  Каждый оставшийся пассажир садится на своё место, если оно свободно,
  и на случайное выбираемое место, если его место уже кем-то занято.

  Чему равна вероятность того, что последний пассажир сядет на своё место? А предпоследний?
\item  Два лекарства испытывали на мужчинах и женщинах. Каждый
человек принимал только одно лекарство. Общий процент людей,
почувствовавших улучшение, больше среди принимавших лекарство А.
Процент мужчин, почувствовавших улучшение, больше среди мужчин, принимавших лекарство В.
Процент женщин, почувствовавших улучшение, больше среди женщин, принимавших лекарство В.

Возможно ли такое?
\item Маша пишет на бумажках два любых различных натуральных числа по своему выбору.
Одну бумажку она прячет в левую руку, а другую — в правую.
Саша выбирает любую Машину руку. Маша показывает число, написанное на выбранной бумажке.
Саша высказывает свою догадку о том, открыл ли он большее из двух чисел или меньшее.
Если Саша не угадал, то Маша выиграла.

Существует ли у Саши стратегия,
гарантирующая ему выигрыш с вероятностью строго более 50\%, даже будучи известной Маше?

\item Илья Муромец стоит на развилке у камня. От камня начинаются
ещё три дороги. Каждая из дорог оканчивается камнем.
И от каждого камня начинаются ещё три дороги. И каждые те три
дороги оканчиваются камнем\ldots И так далее до бесконечности. На
каждой дороге живёт трёхголовый Змей Горыныч. Каждый Змей
Горыныч бодрствует независимо от других с вероятностью одна третья. 

\begin{enumerate}
  \item Какова вероятность того, что ИМ встретит ЗГ, если выбирает дороги равновероятно?
  \item Какова вероятность того, что у ИМ \textit{существует} хотя бы один путь, избегающий встречи с бодрствующими ЗГ?
\end{enumerate}
\item Плоскость разлинована параллельными линиями через каждый сантиметр.
Случайным образом на эту плоскость бросается иголка длины $a<1$.

\begin{enumerate}
\item Какова вероятность того, что иголка пересечёт какую-нибудь линию?
\item Предложите вероятностный способ оценки числа $\pi$
\end{enumerate}

\item В ювелирном магазине широко открыта дверь, а на открытой витрине лежат 150 колец и 420 кулончиков. 
Пока продавщица болтает по телефону проказница-сорока залетает в магазин хватает по одному украшению наугад и относит к себе в гнездо
 до полного опустошения магазина.

Какова вероятность того, что в какой-то момент времени кроме изначального в гнезде сороки колец будет столько же, сколько кулончиков?
\item Десять аргонавтов разного роста высадились на остров и идут по узкой тропинке в случайном порядке.
Каждый аргонавт видит вперёд не далее спины более высокого аргонавта.
Внезапно впереди самого первого аргонавта показалась Медуза Горгона, и все,
кто видел её обратились в камни.

Найдите математическое ожидание числа возникших камней.

\end{enumerate}

\section{Загоночная работа}

Десять тестовых вопросов на интуитивные ловушки по вероятностям. 
Далее играли в ежа, льва, шакала, жирафа и попугая.


\section{todo...}

Уравнение на ожидание



Равновероятные исходы: сложные примеры

Случайные перестановки (заключенные, старушка, а-б-в, старушка два)

Статистика








\newpage

\section{Лог. КЛШ-2022}

\begin{enumerate}
  \item 
\end{enumerate}

В теховском файле \verb|\newpage| стоит, чтобы легко было скопировать секцию, для печати двух копий подряд на одном листе.
Это позволяет экономить бумагу и время при печати :)

\subsection{Плакат}





\Closesolutionfile{solution_file}

% для гиперссылок на условия
% http://tex.stackexchange.com/questions/45415
\renewenvironment{solution}[1]{%
         % add some glue
         \vskip .5cm plus 2cm minus 0.1cm%
         {\bfseries \hyperlink{problem:#1}{#1.}}%
}%
{%
}%



\section{Решения}
\input{all_solutions}


\section{Источники мудрости}

\todo[inline]{передалать потом в bib-файл}

\begin{enumerate}
\item \url{https://github.com/bdemeshev/probability_dna}
\item \url{https://github.com/bdemeshev/probability_pro}
\end{enumerate}

\printbibliography[heading=none]


\end{document}
